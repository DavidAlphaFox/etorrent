\documentclass[a4paper]{memoir}
\usepackage[T1]{fontenc}
\usepackage{charter}
\usepackage{microtype}

\usepackage{amsmath}
\usepackage{amssymb}

\newtheorem{theorem}{Theorem}
\newtheorem{lemma}{Lemma}
\newtheorem{example}{Example}
\newtheorem{remark}{Remark}

\title{eTorrent documentation}
\author{Jesper Louis Andersen \\ jesper.louis.andersen@gmail.com}
\date{\today}
\begin{document}
\maketitle{}
\tableofcontents{}

\section{Introduction}
Why write a complete analysis and documentation on the software? This
is normally not the way Open Source software is written. Rather than
sit down and think it seems most people are happy with writing code
and let the code be documentation of what to do. But the problem with
that approach is that I hop on-and-off on this project. It makes it
impossible to do ad-hoc development.

When you grow up, it becomes clear that writing extensive
documentation for a piece of software gives a much better
result. Documentation is key to correct and elegant
software. Documentation is key to extensive analysis before
code. Thus, we must write documentation.

\chapter{Requirements}
\label{chap:requirements}

The plan is to build a Bittorrent client. What will set this client
off from other clients is the fault-tolerance of the client. In
general, it should not be possible to take the client down, even in
the case of errors in the client, on the disk, system crashes etc.

\paragraph{Fault tolerance} The client should be fault-tolerant. In
general, if any part of the client has an error, it must not result in
the client being brought down. The client should also be able to
recover fast from such an error.

The client should avoid disk-corruption at all costs. It is accepted
if the client assumes the underlying operating system does not corrupt
the disk.

\paragraph{Unattended operation} The client must run unattended. The
interface to the client is a directory hierachy: Torrent files in the
hierachy are downloaded and when a file is removed, it is
stopped. There are no requirement for an interactive interface.

\paragraph{Performance} The client should run at an adequate speed. It
should be able to run up to the usual limits of Disk I/O. The client
should not pursue speed aggressively, use rtorrent for that. The
client must be able to serve a large number of simultaneous
torrents. We aim for thousands of torrents served at the same time, if
the server is large enough. There are 3 iterations: 1 torrent, 100
torrents and 1000 torrents which must be achieved in order in
releases of the software.

\chapter{Analysis}
\section{Pieces}
Pieces are the central elements we exchange between peers. A Torrent
file consists of several pieces. Each are identified by a natural
number indexed at $1$. This identification serves as the primary key
in our implementation. Several informations are linked to the primary
key. First, each piece has a size. This is important, since the last
piece rarely have the size as the other pieces. Second, pieces have
binary data associated with them of the given size. Third, this data
has a checksum. Finally, the piece is mapped into a list of triples,
$(path, offset, length)$ which designates to where the piece should
go.
\begin{example}
  A simple piece could have a list like the following:
\begin{verbatim}
[{foo, 10, 20},
 {bar, 30, 50}]
\end{verbatim}
It should be interpreted as if we should write 20 bytes to
\texttt{foo} at offset 10 and then write 50 bytes to \texttt{bar} at
offset 30.
\end{example}
\begin{remark}
  Invariant: The sum of the sizes in the list of where the piece is
  stored should be equal to the size of the piece.
\end{remark}

\paragraph{On piece size}
It seems correct to keep track of piece size all over the
system. First, if we run several torrents, they may have different
piece sizes. Second, it will greatly reduce the need to take special
care of the last piece.

\paragraph{On piece checksum}
We should always checksum a piece which has been read. First, it alleviates
disk-corruption. A corrupted piece can then never be transmitted over
the network. Second, it is cheap to check a piece in memory. Third, it
serves as a great assertion invariant in the system: All written
pieces should preserve their checksum when read.

When writing a piece, it should be checked as well. There is no
thought in writing something which became accidentally corrupted. As
we mostly retrieve the binary data associated with a piece from a
peer, we really have no control over its correctness, so checked it
must be.

\section{Filesystem interaction}
\subsection{Piece serving}
When we wish to serve a piece from disk, we must carry out a number of
operations: We must locate the piece on disk. We must load it into
memory and we must break it up so it can be sent to the peer who
requested it.

Locating a piece is piece number. If we have the piece
number, we deduce the files which comprises the piece in question and
the (offsets, lengths) we have to read inside them. In other
words, let $pids$ be piece identifications. Further, let $path$ be a
file system path (UNIX notation). Finally, let $offset, length \in
\mathbb{N}$. We have the function:
\begin{equation*}
  \mathtt{locate\_piece} \colon (pid) \to (path, offset, length)\; list
\end{equation*}

Then, when the piece is located, we must load it. Assume the existence
of a function that can read pieces
\begin{equation*}
  \mathtt{read\_piece} \colon (path, offset, length) \; list \to
  binary
\end{equation*}
where $binary$ is binary data. When data has been read, we check
its cryptographic checksum. If the check doesn't match at this point,
we have an error situation which must be handled appropriately.

Then the checksummed piece is sent to the process responsible for peer
communication. Since peers can choose their block size as they see
fit, the cut operation must not be handled centrally, but at the peer
communication process.

\subsection{Piece retrieval}

When we get a piece from a peer, we begin by making a checksum
check. If this check fails, we answer the peer communication process
with an error and note it gave us a bad piece. This can be used by the
piece communication process to mark its peer ``dirty'' and eventually
for disconnecting and blacklisting.

If the piece is ok, we look up the checksum in the map of
checksums. It must match the identification of the piece. If not, it
is an error as well. If both the checksum and identification matches,
we will store the piece.

There are several storage methods available to our disposal:

\paragraph{Method 1}
Create all files. Use the system call \texttt{fseek(3)} to
fast-forward to the point in the file we want to write and write down
the piece at its correct slot.

The advantage of this approach is simplicity. It is easy to
implement. It may introduce sparse files however. We may also pre-fill
all files with an amount of zeros to avoid the sparse file
production. However, this will be a problem because it takes time and
it introduces files on-disk essentially without
information. Pre-filling ensures that the file can always be written
irregardless of free-space however.

We note that the Azureus client seems to be using an approach like
this.

\paragraph{Method 2}
Write the file contigously. Call the on-disk piece locations for
slots. Then we first write to slot 1, then slot 2, then slot 3 and so
forth. Pieces are written as they come in, so they may not be written
in the correct slots in the first place.

This can be alleviated by using a sorting algorithm on the
pieces. There are several applicable sorting algorithms. A simple
solution would be exchanging selection:

Assume the pieces $1$ through $n$ are sorted correctly. We write
pieces contigously to slot $n+1, n+2, \dotsc$. When piece $n+1$ is
retrieved, we exchange the piece in slot $n+1$ with this new piece. To
do this safely, we use a free slot on-disk as a temporary variable and
ensure we copy the piece out of slot $n+1$ first. Thus, a crash will
not result in the loss of data. Note that we then have pieces $1$
through $n+1$ placed correctly. We then run again for slot $n+2$ which
we may have already retrieved. The question is how many exchanges this
makes as disk I/O is pretty heavy and a major limiting factor in
BitTorrent clients.

For a slot there are a maximum of 3 writes: One for the contiguous
write, one when the piece that fits gets written and one for making
place for the fitting piece. Thus, the algorithm is $\mathcal{O}(n)$
with a constant factor of around 3.

The original bittorrent client by Bram Cohen uses a variant of this
approach.

\paragraph{Method 3}
Use \texttt{mmap(2)}. A file is mapped into memory at a given
location. Writes are done to memory and the operating system is
responsible for letting the write go to the disk at the correct
location by the virtual memory subsystem. This is extremely easy. It
is fast as well, but there are a couple of limitations.

In a 32-bit architecture, we don't have enough memory to keep
several gigabytes of data mapped in. Hence, we will either need to use
a pure 64-bit operating system or we will need to devise an algorithm
for mapping parts of files in and out. We need to do this anyway,
since we can't expect to map several file descriptors at the same
time.

Rtorrent is using this approach.

\paragraph{Method 4}
Use internal storage. We can choose to represent the data internally
in a on-disk persistent format. Then, when we have the whole file, we
can write it out. Each piece will get written exactly 2 times, so it
may seem to be better than method number 2. On the other hand, there
are problems with the method: We can't look at data until everything
is downloaded.

\paragraph{Discussion}
My intuition tells me, that method 1 with pre-fill is the easiest to
implement. Thus, we choose to implement that solution first. We can
change to another method later, when the client basics are there and
works.

\subsection{What to do at startup?}
When the system starts, we have no idea of what we have piece-wise of
a torrent. Hence, we must halt all communication with others until we
know what pieces we have and what pieces we miss. We will check one
torrent at a time, which will require some control.

For each torrent, we will begin loading in pieces. Either pieces fail,
or pieces will be checked. If method 2 is chosen for piece storage, we
need to identify read pieces. There must be some error-handling in the
loading code, so we gracefully handle mis-loads.

If a file is missing on disk, we will create it and pre-fill it with
zeros. Hence, we have the following invariant: ``File system processes
can assume there is access to the needed files''.

\subsection{Handling checksum read errors}
What happens when a checksum read reports an error? There are 2 causes
for this: Disk corruption and a system crash/reset. The most probable
is that the system was reset. Thus, we mark the piece as bad and
ignore it as if it did not exist. Done correctly, it seems we can then
continue running.

Disk corruption is much more fatal. We will assume data is not
corrupted on the disk. Modern file systems like ZFS (see \cite{zfs}),
will carry out checks of all read blocks and thus it is near
impossible to have disk corruption in such a scheme.

\section{Peer processes}
\chapter{Programming planning}
\section{Filesystem}
A central problem to the eTorrent project is the File system. The
filesystem processes must be split because the death of one of them
must not take all torrents down. It would rather bad architecture.

\subsection{Processses}
\subsubsection{File process}
For each file which is managed, there is a process which is termed the
``file process''. This process is responsible for managing the file
reads and writes. It has a very simple interface by which it accepts
read and write operations on the file given by byte offset and number
of bytes to read/write. It also contains a timeout for when no-one has
requested any data on the file for some time in which case it closes
down gracefully.

\subsubsection{General idea}
For each torrent, there is a managing proces. This process is
responsible for managing the torrents access to the disk. The
management process is created when a given torrent has been processed
for checksumming and is handed its status upon spawn-time.

When spawned, we get a mapping between piece identifications and the
files we need to read from/write to in order to get the piece loaded
or saved. We use this mapping for lookup in the code.

There are 2 main functions that the management process accepts:
\texttt{read\_piece} and \texttt{write\_piece}. Upon getting a read or
write request the process will look if there is a file process serving
already. If not, it will spawn one and ask it to read/write the data
in question. The process is linked to the file processes, so if any of
these dies, we know it and can act accordingly by cleaning up our map
of files and $pid$s. Since a file process exists when it has done
nothing for some time, it is expected that we will use this feature
quite much.

\subsubsection{File descriptor replacement}
We want a simple algorithm for replacing file descriptors. A very way
which is possible in erlang is to let each file be managed by a
process. This process has a timeout on its main retrieval which will
make it close down if no operations have been served for some time. A
main process will keep track of all file processes and it will also
have an LRU structure for the files. Thus file-descriptor processes can be
purged if some new files has to be opened, but they auto-purge if
no-one uses them.

Ergo, whenever a file process is spawned, the LRU process is informed
about it. It can then ask for a close of a given process if it runs
out of file descriptors.
\begin{remark}
  This is a long term optimization. It should not be implemented in
  the first release.
\end{remark}

\end{document}

%%% Local Variables: 
%%% mode: latex
%%% TeX-master: t
%%% End: 
